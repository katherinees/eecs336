\documentclass[11pt]{article}
\usepackage{amsmath, amsfonts, amsthm}
\usepackage{fancyhdr,parskip}
\usepackage{fullpage}
\usepackage{graphicx}
\usepackage[margin=1.2in]{geometry}
\usepackage[answerdelayed]{exercise}
\usepackage{verbatim}
% \pagenumbering{arabic}

%%
%% Stuff above here is packages that will be used to compile your document.
%% If you've used unusual LaTeX features, you may have to install extra packages by adding them to this list.
%%%%%%%%%%%%%%%%%%%%%%%%%%%%%%%%%%%%%%%%%%%%%%%%%%%%%%%%%%%


\setlength{\headheight}{15.2pt}
\setlength{\headsep}{20pt}
\pagestyle{fancyplain}


%%
%% Stuff above here is layout and formatting.  If you've never used LaTeX before, you probably don't need to change any of it.
%% Later, you can learn how it all works and adjust it to your liking, or write your own formatting code.
%%%%%%%%%%%%%%%%%%%%%%%%%%%%%%%%%%%%%%%%%%%%%%%%%%%%%%


%%%%%%%%%%%%%%%%%%%%%%%%%%%%%%%%%%%%%%%%%%%
%% This section contains some useful macros that will save you time typing.
%%

% Using \displaystyle (or \ds) in a block of math has a number of effects, but most notably, it makes your fractions come out bigger.
\newcommand{\ds}{\displaystyle}

% These lines are for displaying integrals; typing \dx will make the dx at the end of the integral look better.
\newcommand{\is}{\hspace{2pt}}
\newcommand{\dx}{\is dx}

% These commands produce the fancy Z (for the integers) and other letters conveniently.
\newcommand{\Z}{\mathbb{Z}}
\newcommand{\Q}{\mathbb{Q}}
\newcommand{\R}{\mathbb{R}}
\newcommand{\C}{\mathbb{C}}
\newcommand{\F}{\mathbb{F}}

\newcommand{\irr}{\operatorname{irr}}

\newcommand{\frob}{\operatorname{Frob}}
\newcommand{\aut}{\operatorname{Aut}}
\newcommand{\maxp}{\textrm{MaxProfit}}

%%%%%%%%%%%%%%%%%%%%%%%%%%%%%%%%%%%%%%%%%%%%%%%%%%


%%%%%%%%%%%%%%%%%%%%%%%%%%%%%%%%%%%%%%%%%%%%%%
%% This is the header.  It will appear on every page, and it's a good place to put your name, the assignment title, and stuff like that.
%% I usually leave the center header blank to avoid clutter.
%%

\fancyhead[L]{EECS3 336 Homework 5}
\fancyhead[C]{\empty}
\fancyhead[R]{Katherine Steiner}

%%%%%%%%%%%%%%%%%%%%%%%%%%%%%%%%%%%%%%%%%%%%%%%%


\begin{document}
We want to design an algorithm for a cryptocurrency startup. We're given two arrays of positive number $N=\{N_1, \dots, N_T\}$ and $W=\{W_1, \dots, W_T\}$ which represent the profits from NorthCoins and WestCoins respectively. Each type of coins requires its own software which takes 1 unit of time to load. 
\begin{enumerate}
\item First we define several subproblems. Let $\textrm{ProfitN}(t)$ be the subproblem where the supercomputer mines optimally for time intervals 1 to $t-1$, and mines NorthCoins at time $t$. Likewise define $\textrm{ProfitW}(t)$ and $\textrm{ProfitLoad}(t)$ so that the computer mines WestCoins or loads software at time $t$, respectively. We order subproblems by increasing $t$. Let $\textrm{MaxProfit}(t)$ be the profit if the computer mines optimally for time intervals $1$ to $t-1$. 
\item For $t=1$, then $\textrm{MaxProfit}(1) = \max(N_1, W_1)$. 

For $t=2$, $\textrm{MaxProfit}(2) = \max(N_1+N_2, W_1+W_2)$. 
\item The recurrence relation

Let $\textrm{ProfitSwitch}(t)$ be a subproblem where the computer is mining either 
\[ \textrm{MaxProfit}(t) = \max(\textrm{MaxProfit}(t-2)+N_t, \maxp(t-2)+W_t, \maxp(t-1)\]
\item We prove that the recurrence relation is correct
\item An algorithm for finding a solution to the subproblem
\item An algorithm for finding the optimal solution to the original problem
\item The running time. 
\end{enumerate}
\end{document}

























