\documentclass[11pt]{article}
\usepackage{amsmath, amsfonts, amsthm}
\usepackage{fancyhdr,parskip}
\usepackage{fullpage}
\usepackage{graphicx}
\usepackage[margin=1.2in]{geometry}
\usepackage[answerdelayed]{exercise}
\usepackage{verbatim}
% \pagenumbering{arabic}

%%
%% Stuff above here is packages that will be used to compile your document.
%% If you've used unusual LaTeX features, you may have to install extra packages by adding them to this list.
%%%%%%%%%%%%%%%%%%%%%%%%%%%%%%%%%%%%%%%%%%%%%%%%%%%%%%%%%%%


\setlength{\headheight}{15.2pt}
\setlength{\headsep}{20pt}
\pagestyle{fancyplain}


%%
%% Stuff above here is layout and formatting.  If you've never used LaTeX before, you probably don't need to change any of it.
%% Later, you can learn how it all works and adjust it to your liking, or write your own formatting code.
%%%%%%%%%%%%%%%%%%%%%%%%%%%%%%%%%%%%%%%%%%%%%%%%%%%%%%


%%%%%%%%%%%%%%%%%%%%%%%%%%%%%%%%%%%%%%%%%%%
%% This section contains some useful macros that will save you time typing.
%%

% Using \displaystyle (or \ds) in a block of math has a number of effects, but most notably, it makes your fractions come out bigger.
\newcommand{\ds}{\displaystyle}

% These lines are for displaying integrals; typing \dx will make the dx at the end of the integral look better.
\newcommand{\is}{\hspace{2pt}}
\newcommand{\dx}{\is dx}

% These commands produce the fancy Z (for the integers) and other letters conveniently.
\newcommand{\Z}{\mathbb{Z}}
\newcommand{\Q}{\mathbb{Q}}
\newcommand{\R}{\mathbb{R}}
\newcommand{\C}{\mathbb{C}}
\newcommand{\F}{\mathbb{F}}

\newcommand{\irr}{\operatorname{irr}}

\newcommand{\frob}{\operatorname{Frob}}
\newcommand{\aut}{\operatorname{Aut}}
\newcommand{\pn}{\textrm{ProfitN}}
\newcommand{\pw}{\textrm{ProfitW}}
\newcommand{\pl}{\textrm{ProfitL}}
\newcommand{\opt}{\textrm{Opt}}

%%%%%%%%%%%%%%%%%%%%%%%%%%%%%%%%%%%%%%%%%%%%%%%%%%


%%%%%%%%%%%%%%%%%%%%%%%%%%%%%%%%%%%%%%%%%%%%%%
%% This is the header.  It will appear on every page, and it's a good place to put your name, the assignment title, and stuff like that.
%% I usually leave the center header blank to avoid clutter.
%%

\fancyhead[L]{EECS3 336 Homework 5}
\fancyhead[C]{\empty}
\fancyhead[R]{Katherine Steiner}

%%%%%%%%%%%%%%%%%%%%%%%%%%%%%%%%%%%%%%%%%%%%%%%%


\begin{document}
We want to design an algorithm for a cryptocurrency startup. We're given two arrays of positive number $N=\{N_1, \dots, N_T\}$ and $W=\{W_1, \dots, W_T\}$ which represent the profits from NorthCoins and WestCoins respectively. Each type of coins requires its own software which takes 1 unit of time to load. 
\begin{enumerate}
\item First we define several subproblems. Let $\pn(t)$ be the subproblem where the computer mines optimally for time intervals up to $t-1$, and mines NorthCoins at time $t$. Likewise define $\textrm{ProfitW}(t)$ and $\pl(t)$ so that the computer mines WestCoins or loads software at time $t$, respectively. We order subproblems by increasing $t$.
%Let $\textrm{ProfitN}(t)$ be the subproblem where the supercomputer mines optimally for time intervals 1 to $t-1$, and mines NorthCoins at time $t$. Likewise define $\textrm{ProfitW}(t)$ and $\textrm{ProfitLoad}(t)$ so that the computer mines WestCoins or loads software at time $t$, respectively. We order subproblems by increasing $t$. Let $\textrm{MaxProfit}(t)$ be the profit if the computer mines optimally for time intervals $1$ to $t-1$. 
\item The base cases are as follows: $\pn(1) = N_1$, $\pw(1) = W_1$, and $\pl(1) = 0$. 
\item The recurrence relations are as follows:
\[ \pn(t+1) = N_{t+1} + \max(\pn(t), \pl(t)),\]
\[ \pw(t+1) = W_{t+1} + \max(\pw(t), \pl(t)),\]
\[ \pl(t+1) = \max(\pn(t), \pw(t)).\]
In words, $\pn(t+1)$ is the maximum profit such that the computer \textit{must} mine a NorthCoin at time $t+1$ (hence the $N_{t+1}$ summand) and mines optimally before that (at time $t$, it either mined a NorthCoin or loaded software to be able to mine NorthCoins, whichever is better, hence the second summand). $\pw(t+1)$ is analogous. 

$\pl(t+1)$ is the maximum profit such that the computer \textit{must} load software at time $t+1$ and mines optimally before that. Since it makes no sense for the computer to load software for two consecutive time intervals, the computer must have either mined NorthCoins or WestCoins at time $t$, and therefore we say that it did whichever gives a higher profit. 
\item We prove that the recurrence relations are correct.
\begin{proof} We claim that the recurrence relation for $\pn(t)$ is correct. 
Consider $\pn(i)$ for some $i$. By assumption, the machine is mining optimally for time $1$ to $i-1$ and must mine NorthCoins at time $i$. We have two cases (these are the only two cases, since these are the only possibilities which allow for feasible solutions where our algorithm solution can mine NorthCoins at time $i$):

Case 1. Suppose the algorithm's machine mined NorthCoins at time $i-1$. Then in the optimal solution, the machine also mined NorthCoins at time $i-1$. It follows that in the optimal solution, we should mine NorthCoins at time $i$ (loading software won't improve our profit, and mining WestCoins is impossible because we have the wrong software). Thus, we get that
\[ \opt(i) = N_i + \pn(i-1).\]

Case 2. Suppose the algorithm's machine loaded NorthCoin software at time $i-1$. Then in the optimal solution, the machine also loaded NorthCoin software at time $i-1$. It follows that in the optimal solution, we should mine NorthCoins at time $i$ (loading software again won't improve our profit, and we don't have the right software to mine WestCoins). Thus, we get that
\[ \opt(i) = N_i + \pl(i-1).\]

These are all possible cases. In each case, we've shown what the optimal solution must be. Combining cases, we find that in general, \[\opt(i) = N_i + \max(\pn(i-1), \pl(i-1)).\] 
Rewriting indices, we observe that this is exactly the recurrence relation for $\pn(t)$:
\[ \pn(t+1) = N_{t+1} + \max(\pn(t), \pl(t)).\]
The recurrence relation is therefore correct for $\pn(t)$. 
\end{proof}
\begin{proof} We claim that the recurrence relation for $\pw(t)$ is correct. The argument is exactly the same as for the recurrence relation for $\pn(t)$. \end{proof}
\begin{proof}
We claim that the recurrence relation for $\pl(t)$ is correct. Consider $\pl(i)$ for some $i$. By assumption, the machine is mining optimally for time $1$ to $i-1$ and must load software at time $i$. Noting that it doesn't make sense for the machine to load software for two consecutive time intervals, then the algorithm machine must have mined either NorthCoins or WestCoins at time $i-1$. But we assumed that the algorithm machine was mining coins optimally at time $i-1$. So the algorithm machine mined NorthCoins if $\pn(i-1) > \pw(i-1)$ and WestCoins otherwise. If at some point $k$, the optimal solution loads software, it follows that at time $k-1$, the optimal machine mined NorthCoins if $\pn(k-1) > \pw(k-1)$ and WestCoins otherwise. Observing that these relationships are the same, it follows that the recurrence relation
\[ \pl(t+1) = \max(\pn(t), \pw(t))\]
is correct. 
\end{proof}
%Suppose in the optimal solution, the computer mines NorthCoins at time $t-1$. Now, by assumption, the computer which gives the profit $\pn(t)$ mines optimally at time $t-1$ and mines NorthCoins at time $t$. Therefore, 
%
%By assumption, $\pn(t+1)$ has the machine mining optimally through time $t$. Then $\pn(t-1) = \opt(t-1)$. 
%
%Let $\opt(t+1)$ be the optimal profit at time $t+1$. 
%
%We want to show that $\pn(t)$ follows the recurrence relation defined above. 
\item An algorithm for finding a solution to the subproblem is as follows: for given time $t$, calculate $\pn(t)$, $\pw(t)$, and $\pl(t)$. The overall optimal profit is \[\max(\pn(t), \pw(t), \pl(t)).\]
\item An algorithm for finding the optimal solution to the original problem is as follows:

If $T = 1$, then the overall maximum profit is just $\max(N_1, W_1)$. 

Otherwise, create arrays $\pn$, $\pw$, and $\pl$. Initialize each according to the base cases from (2): let $\pn[0] = N_1$, $\pw[0] = W_1$, and $\pl[0] = 0$. For each time interval, fill in the next element in each of these arrays according to the recurrence relations in (3). The overall maximum profit is $\max(\pn(T), \pw(T))$. In pseudocode, 

\begin{verbatim}
if size of north is 1, return max(north[0], west[0])

PN = [north[0]]
PW = [west[0]]
PL = [0]

for i = 0 to T-1:
	PN[i+1] = north[i+1]+max(PN[i], PL[i])
	PW[i+1] = west[i+1]+max(PW[i], PL[i])
	PL[i+1] = max(PN[i], PW[i])

return max(PN[T], PW[T])
\end{verbatim}

\item The \verb|for| loop runs $T$ times, and the operations inside the loop take $O(1)$ time to complete. The time complexity is therefore $O(T)$, linear time. 
\end{enumerate}
\end{document}

























